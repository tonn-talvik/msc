\documentclass[a4paper]{article}

\usepackage[utf8]{inputenc}
\usepackage{a4wide}


\begin{document}
\title{Magistritöö probleemipüstitus}
\author{
  Talvik, Tõnn\\
  \texttt{talvikster@gmail.com}
  \and
  Juhendaja: Uustalu, Tarmo
}
\date{Mai 2016}
\maketitle

\section{Võimalik töö pealkiri}
Efektide analüüside ja nendel põhinevate programmiteisenduste sertifitseerimine

\section{Uurimistöö küsimused -- mida tahame teada?}

\begin{itemize}
\item Kas efektide analüüsid ja optimisatsioonid toimivad keele juures, mis toetab andmetüüpe (siin naturaalarvude näide).
\item Kas Agda-taolises eksperimentaalses keeles on mõistliku vaevaga realiseeritav idee tõendamise taseme raamistu efektide analüüsiks ja nendele põhinevateks programmiteisendusteks.
\end{itemize}


\section{Uurimistöö objekt -- mille kohta?}

Agdas formaliseeritav näitekeel -- baaskeelena tüübitud $\lambda$-arvutus pluss tõeväärtused ja naturaalarvud, ning selle laiendused paari efektiga (nt erandid, mittedeterminism, muteeritav olek).

\section{Olulised mõisted ja teooriad}
Lähtematerjaliks on N. Bentoni ja kaasautorite artiklid.

\section{Uurimistöö eesmärk: miks me tahame teada?}
Sedalaadi ülesande realisatsioon Agdas, mis iseenesest on eksperimentaalne keel, on uudne. Teoreetilisel tasemel on uudne, aga mitte liiga keeruline või liiga suur üllatus, et efektide analüüsid ja optimisatsioonid toimivad keele juures, mis toetab andmetüüpe.

\section{Uurimistöö kavand: kuidas vastame küsimustele?}

\begin{itemize}
\item Formaliseerida näitekeel Agdas: tüübid, tüübitud avaldised, tüüpide ja avaldiste semantika.
\item Realiseerida efektid: erandid, mittedeterminism, muteeritav olek.
\item Realiseeritud efektide jaoks formaliseerida mõned kvantitatiivsed analüüsid, näidata, et need on korrektsed.
\item Realiseeritud analüüside jaoks formaliseerida mõned programmiteisendused, näidata, et need on korrektsed.
\end{itemize}

\end{document}
