\documentclass[a4paper,12pt]{article}
\usepackage[lmargin=30mm,rmargin=30mm,tmargin=25mm,bmargin=25mm]{geometry}
\usepackage{mathptmx}% http://ctan.org/pkg/mathptmx

%\usepackage{ucs}
\usepackage[utf8x]{inputenc}

%\usepackage{amssymb}
%\usepackage{bbm}
%\usepackage[greek,english]{babel}
%\usepackage{autofe}



\usepackage{fancyhdr}
\fancypagestyle{FirstPage}{\fancyhf{}\renewcommand{\headrulewidth}{0pt}\fancyfoot[C]{Tallinn 2017}}

\renewcommand*\contentsname{\hfil Sisukord\hfil}

\usepackage{biblatex}
\bibliography{thesis}


%\linespread{1.4}
\renewcommand{\baselinestretch}{1.5}

\setlength{\parindent}{0pt}
\setlength{\parskip}{12pt}

\usepackage{titlesec}

% http://tex.stackexchange.com/questions/299969/titlesec-loss-of-section-numbering-with-the-new-update-2016-03-15
\usepackage{etoolbox}
\makeatletter
\patchcmd{\ttlh@hang}{\parindent\z@}{\parindent\z@\leavevmode}{}{}
\patchcmd{\ttlh@hang}{\noindent}{}{}{}
\makeatother

\titlespacing*{\section}{0pt}{60pt}{18pt}
\titlespacing*{\subsection}{0pt}{60pt}{18pt}

\usepackage{listings}
\lstdefinelanguage{agda}{
  morekeywords={abstract,constructor,data,field,
    forall,hiding,import,in,infix,infixl,infixr,let,
    module,mutual,open,postulate,primitive,Prop,
    private,public,quoteGoal,quoteTerm,quote,record,
    renaming,rewrite,Set,syntax,unquote,using,where,with},%
  keywordstyle=\bfseries,
  comment=[l]{--},
  morecomment=[s]{\{-}{-\}}
}


\lstset{language=agda,basicstyle=\linespread{0.8}}
%\newfontfamily\listingsfont[Scale=.7]{DejaVu Sans Mono}
%\lstset{basicstyle=\listingsfont}

\usepackage{float}
\floatstyle{boxed} 
\restylefloat{figure}

\usepackage{caption}
\captionsetup[figure]{name=Joonis}


\begin{document}


% ------------------------------------------------------------
  \begin{center}
    \textsc{tallinna tehnikaülikool}\\
    Infotehnoloogia teaduskond\\
    Arvutiteaduse instituut\\
    \vfill
    
    Tõnn Talvik 132619IAPM
    \vskip 4em 
    \LARGE\textsc{efekti analüüside ja nendel põhinevate programmiteisenduste sertifitseerimine}\\
    \normalsize\vskip 4em 
    Magistritöö\\
    \vskip 4em    
  \end{center}
  
  \begin{flushright}
    \begin{tabular}{ r l }
      Juhendaja:& Tarmo Uustalu\\
      & Professor \\
    \end{tabular}
    \vfill
  \end{flushright}
  
  \thispagestyle{FirstPage}
  \clearpage
  \setcounter{page}{2}
% ------------------------------------------------------------

\section*{\begin{center}Autorideklaratsioon\end{center}}

Kinnitan, et olen koostanud antud lõputöö iseseisvalt ning seda ei ole kellegi teise poolt varem kaitsmisele esitatud. Kõik töö koostamisel kasutatud teiste autorite tööd, olulised seisukohad, kirjandusallikatest ja mujalt pärinevad andmed on töös viidatud.

Autor: Tõnn Talvik

8. mai 2017
\clearpage

\section*{\begin{center}Annotatsioon\end{center}}

[tekst]
         
Lõputöö on kirjutatud eesti keeles ning sisaldab teksti [lehekülgede arv töö põhiosas] leheküljel, [peatükkide arv] peatükki, [jooniste arv] joonist, [tabelite arv] tabelit.
\clearpage

\section*{\hfil Abstract\hfil\\
\begin{center}Certification of effect analysis and program transformations based on the analysis\end{center}}
[text]
The thesis is in Estonian and contains [pages] pages of text, [chapters] chapters, [figures] figures, [tables] tables.
\clearpage

\tableofcontents
\clearpage

\listoffigures
\clearpage







\section{Sissejuhatus}

Sissejuhatuses tutvustab autor töö teemat, töö eesmärke, lahendatavat probleemistikku,
andes samuti ülevaate töö ülesehitusest. Sissejuhatuses kirjeldatakse ka töö
lähtetingimused, alamülesanded ja vajadusel ka täiendavad nõuded (vt jaotist 2.4 ).
This is obvious \cite{Benton2016}. \cite{Katsumata2014}

Lõputöös peab sisalduma selge lõpetaja poolt lahendatava ülesande püstitus.

Magistritöös esitatakse lahendatava ülesande püstitus töö sissejuhatuses, kattes
järgmised punktid:
- töös lahendatavad küsimused ja lähtetingimused,
- eritingimused, mida on rakendatud ülesande lahendamisel/ülesande püstitamisel.
\clearpage

\section{Eranditega keel}
aoe ushoae suhtoase hutasone hutsanohtinahtoisanht

\subsection{Erandite gradeering}
asoe hut

\begin{figure}
  \begin{lstlisting}
  data Exc : Set where
    err : Exc
    ok : Exc -- kommentaar
    errok : Exc

  \end{lstlisting}
  \caption{Erandite andmetüübi konstruktorid.}
\end{figure}

\subsection{Tüübituletus}
aoeu
\subsection{Semantika}
aoeu
\subsection{Optimisatsioonid}
aeoust haoseu th

\clearpage

\section{Mitte-deterministlik keel}
aseo huasousato usaoheu s

\section{Kokkuvõte}
Kokkuvõttes esitab autor töö põhieesmärgi, vastused sissejuhatuses püstitatud
küsimustele, toob välja töö olulisemad tulemused ja järeldused.


\renewcommand{\baselinestretch}{1.15}
\printbibliography[title={Kasutatud kirjandus}]

\end{document}
